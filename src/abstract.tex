\begin{otherlanguage}{magyar}

  \paragraph*{Kivonat}
  \phantomsection
  \addcontentsline{toc}{chapter}{Kivonat}
  \thispagestyle{plain}
  {
  \selecthungarian

  Egyre több, egyre inkább komplex szoftver vesz körül minket, sok esetben ezek akár részét képezhetik kritikus rendszereknek is. Az ilyen rendszerek fő jellemzője, hogy a legapróbb hibáik is komoly következményekkel járhatnak. A forráskód statikus analízise egy, a kritikus szoftverrendszereknél általánosan elfogadott megközelítés, amely a hibák mihamarabbi megtalálását célozza meg.  A statikus analízis már a fejlesztési folyamat korai szakaszaiban is alkalmazható, mivel nincs szükség a kód fordítására és futtatására az ellenőrzés véghezviteléhez. A megközelítést számos eszköz megvalósítja, amelyek képesek visszajelzést adni a potenciális hibahelyeken túl arról is, hogy a forráskód megfelel-e a kódolási szabályoknak és követelményeknek.

  Habár több statikus analízis eszköz is elérhető generikus nyelvek elemzéséhez, és ezek többsége beépíthető a folytonos integráció folyamatába is, JavaScript esetén ez nem mondható el annak dinamikus volta végett. A dinamikus nyelvek sajátosságai miatt csak pár eszköz érhető el JavaScript forráskódok kódtárszintű statikus analíziséhez, illetve az eddig ismert ilyen eszközök nem nyújtanak egyszerre megoldást alaki ellenőrzésre, futási utak meghatározására és folytonos integrációba történő integrációra.
  
  Jelen dolgozatomban egy olyan, a folytonos integráció kiegészítésére képes keretrendszert tervezek, valósítok meg és értékelek, amely képes nagyméretű és gyakran változó JavaScript forráskódtárak konfigurálható statikus analízisére. A keretrendszer alapjául szolgáló újszerű megközelítésnek köszönhetően az eddig megszokott megoldások helyett a felhasználók egyszerűbb módon fejezhetik ki az ellenőrzésre szánt követelményeket és képesek a több forráskódon átívelő követelményeket hatékonyabban elemezni.

  }

\end{otherlanguage}

\cleardoublepage

\paragraph*{Abstract}
\phantomsection
\addcontentsline{toc}{chapter}{Abstract}
\thispagestyle{plain}

We are surrounded by more and more complex software that operate in mission-critical systems. Even small errors in these software can lead to consequences ranging from unpleasant to expensive. A proven approach for detecting mistakes early in the development cycle is static analysis of the source code. Static analysis tools report whether the software conforms to the coding rules and requirements without executing the code itself.

While multiple static analysis tools exist for general purpose programming languages and these are generally part of the continuous integration systems, this is not the case with JavaScript. Because of the untyped nature of this dynamic language there are only a few tools that perform static analysis on JavaScript code repositories on a global level.

In this report I design, implement and evaluate a framework aiming to extend the continuous integration workflow of large JavaScript repositories with various static analysis tools and techniques.

\clearpage
