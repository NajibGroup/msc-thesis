\begin{otherlanguage}{magyar}

  \paragraph*{Kivonat}
  \phantomsection
  \addcontentsline{toc}{chapter}{Kivonat}
  \thispagestyle{plain}
  {
  \selecthungarian

  Egyre több, egyre inkább komplex szoftver vesz körül minket, amelyek gyakran kritikus rendszereket vezérelnek. Az ilyen rendszerek fő jellemzője, hogy a legapróbb hibáik is komoly következményekkel járhatnak. A forráskód statikus analízise egy, a kritikus szoftverrendszereknél általánosan elfogadott megközelítés, amely a hibák mihamarabbi megtalálását célozza meg. A statikus analízis már a fejlesztési folyamat korai szakaszaiban is alkalmazható, mivel nincs szükség a kód fordítására és futtatására az ellenőrzés véghezviteléhez. A megközelítést számos eszköz megvalósítja, amelyek képesek visszajelzést adni a potenciális hibahelyeken túl arról is, hogy a forráskód megfelel-e a kódolási szabályoknak és követelményeknek.

  Habár több statikus analízis eszköz is elérhető általános célú nyelvek elemzéséhez, és ezek gyakran a folytonos integráció részét képzik, JavaScript esetén ez nem mondható el annak dinamikus jellege miatt. A dinamikusan tipizált nyelvek sajátosságai miatt csak pár eszköz érhető el JavaScript forráskódok kódtárszintű statikus analíziséhez, illetve az eddig ismert ilyen eszközök nem nyújtanak egyszerre megoldást alaki és globális ellenőrzésre, futási utak meghatározására és folytonos integrációval történő alkalmazásra.

  Jelen dolgozatomban egy olyan, a folytonos integráció kiegészítésére képes keretrendszert tervezek, valósítok meg és értékelek, amely képes nagyméretű és gyakran változó Java\-Script forráskódtárak konfigurálható statikus analízisére. A keretrendszer alapjául szolgáló újszerű megközelítésnek köszönhetően az eddig megszokott megoldások helyett a felhasználók egyszerűbb módon fejezhetik ki az ellenőrzésre szánt követelményeket és képesek a több forráskódon átívelő követelményeket hatékonyabban ellenőrizni.

  }

\end{otherlanguage}

\cleardoublepage

\paragraph*{Abstract}
\phantomsection
\addcontentsline{toc}{chapter}{Abstract}
\thispagestyle{plain}

We are surrounded by more and more complex software that operate in mission-critical systems. Even small errors in these software can lead to serious consequences that may be too expensive to let happen. Static analysis is a proven approach for detecting mistakes in the source code early in the development cycle. Since static analysis does not compile or run the code, it can be applied at an early state of development. With static analysis it is possible to check whether the software conforms to the coding rules and requirements, and to locate potential errors.

While multiple static analysis tools exist for general purpose programming languages and these are generally part of the continuous integration systems, this is not the case with JavaScript. Due to the dynamically typed nature of this language there are only a few tools available for JavaScript codebases. Also, there are currently no tools available jointly providing lower level and global static analysis, finding control flows, and providing integration points for continuous integration systems.

In this thesis I design, implement and evaluate a framework extending the continuous integration workflow of large and frequently changing JavaScript repositories with configurable static analysis tools and techniques. Due to the novel approach of the framework, its users can express requirements easier and they are able to check global level requirements more efficiently.

\clearpage
