% !TEX root = ../main.tex
\chapter{Background and Related Work}
\label{chap:background-and-related-work}

In this chapter I enumerate a subset of similar systems, approaches and discuss related work.

\section{Tern}
\textquote[http://ternjs.net]{Tern is a stand-alone code-analysis engine for JavaScript. It is intended to be used with a code editor plugin to enhance the editor's support for intelligent JavaScript editing. Features provided are:

\begin{itemize}[topsep=0pt]
	\item Autocompletion on variables and properties
	\item Function argument hints
	\item Querying the type of an expression
	\item Finding the definition of something
	\item Automatic refactoring
\end{itemize}

Tern is open-source (MIT license), written in JavaScript, and capable of running both on node.js and in the browser.}

The Tern suite is a modular, extendable stand-alone system. Editor plugins communicate with the Tern server module, connected to the Acorn parser (introduced in~\Cref{sect:acorn}) and the inference engine. Third-party plugins can introduce implementation environmental or behavioral information for the system, for example ECMAScript module loading rules, or node.js~\cite{nodejs} specific variables.~\cite{tern-docs}


\section{TAJS}
Type Analyzer for JavaScript (TAJS)~\cite{tajs} is a dataflow analysis tool infering type information and call graphs. The current version (as of 2016) can model scripts of ECMAScript 3; it also contains model of the standard library and partial model of the HTML DOM and browser API.~\cite{tajs-git}

The initial aim of TAJS was to warn programmers for the following problematic cases. This enumeration follows~\cite{jensen_type_2009}.

\begin{itemize}[topsep=0pt]
  \item invoking a non-function value as a function
  \item reading an absent variable
  \item accessing a property of \code{null} or \code{undefined}
  \item reading an absent property of an object
  \item writing to variables or object properties that are never read
  \item implicitly converting a primitive value to an object
  \item implicitly converting \code{undefined} to a number
  \item calling a function object both as a function and as a constructor or passing function parameters with varying types
  \item calling a built-in function with an invalid number of parameters or with a parameter of an unexpected type
\end{itemize}


\section{TRICORDER}
TRICORDER~\cite{tricorder} is a pluggable program analysis platform used internally at Google, helping developers and reviewers notice possible problems with code changes. The system mainly supports C$++$, Go, and Java codes, but it has support for JavaScript too.

Related researches show that static analysis tools are either not used or ignored, when not configured correctly and take more time from the user than necessary. \textquote[\cite{tricorder}]{High false positive rates, confusing output, and poor integration into the developers' workflow all contribute to the lack of use in everyday development activities \cite{johnson2013don, layman2007toward}.

TRICORDER introduces an effective place to show warnings. Given that all developers at Google use code review tools before submitting changes, TRICORDER's primary use is to provide analysis results at code review time. This has the added benefit of enabling peer accountability, where the reviewer will see if the author chose to ignore analysis results.}


\section{JavaScript Parsers}
\label{sect:javascript-parsers}
In this section I showcase the most used, \emph{trending} JavaScript parser technologies and justify why I have chosen the Shape Security Shift family as the parser and additional toolset for my approach.

\subsection{Acorn}
\label{sect:acorn}
Acorn~\cite{acorn} is an open-source, small JavaScript written in ES6. It is up-to-date, able to parse ECMAScript version 3, 5, 6, 7, and the newest one, 8. The resulting AST structure of the parser conforms the ESTree specification~\cite{estree}.

It is also able to parse multiple files into a single AST, connected with a \code{Program} node. This is not to be confused with an ASG connecting multiple ASTs. To analyze and navigate in the resulting AST, Acorn provides a walker interface, to be used with a visitor pattern based algorithm.

The parser is also configurable with several options, \code{locations} being one of the most useful for my approach. Setting this option stores the location of the represented source snippet in the AST node. There is also an error-tolerant version of the parser enabling parsing unfinished or syntactically incorrect sources.

\textquote[\cite{acorn}]{Acorn is designed support allow plugins which, within reasonable bounds, redefine the way the parser works. Plugins can add new token types and new tokenizer contexts (if necessary), and extend methods in the parser object.}

\subsection{Esprima}
Esprima is also an open-source, ECMAScript standard-compliant parser. It fully supports ES7, and produces ESTree models. It has a great user-base and several tools depend on it. It also has experimental support for JSX, an XML syntax extending JavaScript for React~\cite{react}, \textquote[\cite{react-tutorial}]{a declarative, efficient, and flexible JavaScript library for building user interfaces}.


\subsection{Shift}
\label{sect:shift}
The Shift~\cite{shift} family consists of several tools. Besides the parser, there is a scope analyzer, a code validator, fuzzer, and a code generator, besides others.

\subsubsection{Shift AST}
The reason behind the number of tooling in this family is due to the fact that Shift does not conform the ESTree specification. In 2014, Shape Security, the company behind Shift announced a new JavaScript AST specification~\cite{shift-spec}. This specification was developed with ECMAScript 6 in mind, along with analysis and transformation.

The specification describes interface for an AST syntax that can represent the structure of an ECMAScript source code. According to Shape Security, a \textquote[\cite{shift-spec-comparison}]{good AST format…

\begin{itemize}
	\item minimizes the number of inhabitants that do not represent a program.
	\item is at least partially homogenous to allow for a simple and efficient visitor.
	\item does not impede moving, copying, or replacing subtrees.
	\item discourages duplication in code that operates on it.
\end{itemize}
}

\subsubsection{Shift Scope Analyzer}
\textquote[\cite{shift-scope}]{The Shift Scope Analyser produces a data structure called a scope tree that represents all of the scoping information of a given program. Each element of the scope tree represents a single scope in the analysed program, and contains many pieces of information, including:

\begin{itemize}[topsep=0pt]
	\item the scope type (there are 12 of them!)
	\item the AST node associated with the scope
	\item variables declared within that scope, each of which points to its declarations and references
	\item whether the scope contains a with statement or direct call to eval, making it dynamic
\end{itemize}}

\subsubsection{Additional Notes}
\begin{itemize}[topsep=0pt]
	\item Besides JavaScript, most of the Shift family is also available for Java projects. This makes it easier to integrate it with projects and tools only available for Java.

	\item Shape Security has a project, Bandolier~\cite{bandolier} for packaging projects with ES6 modules into a single JavaScript file, imitating the export-import mechanism, related to my approach.

	\item Shape Security is also developing a semantic transformer~\cite{shift-semantics-java} for ECMAScript ASTs, also related to my approach.
\end{itemize}

\subsection{Comparison of Parser Technologies}
In order to find a best parser and related technologies, I had to compare them: measure their speed, investigate their parameters and output model, and transform their extra functions into potential features of my approach.

\subsubsection{Speed} Although speed is not the most important property of a system aiming to make sure no errors are present, swift response can boost the performance of the user. \Cref{table:speed-comparison-of-parsers} shows the time difference between parsers processing various source codes repositories. The benchmark~\footnote{\url{http://esprima.org/test/compare.html}} was run on a personal computer without specific considerations or fine-tuning for benchmarks. Its sole purpose is to get a rough comparison between the different technologies available.

It is visible that Shift NEE\footnote{Early error checking disabled. NEE -- No Early Errors} is one of the fastest parsers available.

\begin{table}[!htb]
\centering
\begin{tabular}{@{}lllllll@{}}
\toprule
\textbf{Source}                                               & \textbf{\begin{tabular}[c]{@{}l@{}}Esprima\\ 2.7.2\end{tabular}} & \textbf{UglifyJS2}                                      & \textbf{Traceur}                                        & \textbf{\begin{tabular}[c]{@{}l@{}}Acorn\\ 2.4.0\end{tabular}} & \textbf{Shift}                                          & \textbf{\begin{tabular}[c]{@{}l@{}}Shift\\ (NEE)\end{tabular}} \\ \midrule
\begin{tabular}[c]{@{}l@{}}jQuery.Mobile\\ 1.4.2\end{tabular} & \begin{tabular}[c]{@{}l@{}}154.0\\ ±22.3\%\end{tabular}          & \begin{tabular}[c]{@{}l@{}}244.6\\ ±8.4\%\end{tabular}  & \begin{tabular}[c]{@{}l@{}}304.6\\ ±15.1\%\end{tabular} & \begin{tabular}[c]{@{}l@{}}215.3\\ ±16.9\%\end{tabular}        & \begin{tabular}[c]{@{}l@{}}480.7\\ ±13.1\%\end{tabular} & \begin{tabular}[c]{@{}l@{}}119.9\\ ±11.9\%\end{tabular}                    \\
\begin{tabular}[c]{@{}l@{}}Angular\\ 1.2.5\end{tabular}       & \begin{tabular}[c]{@{}l@{}}125.5\\ ±16.3\%\end{tabular}          & \begin{tabular}[c]{@{}l@{}}212.2\\ ±11.2\%\end{tabular} & \begin{tabular}[c]{@{}l@{}}254.1\\ ±20.7\%\end{tabular} & \begin{tabular}[c]{@{}l@{}}146.3\\ ±18.6\%\end{tabular}        & \begin{tabular}[c]{@{}l@{}}452.7\\ ±12.5\%\end{tabular} & \begin{tabular}[c]{@{}l@{}}94.6\\ ±18.2\%\end{tabular}                     \\
\begin{tabular}[c]{@{}l@{}}React\\ 0.13.3\end{tabular}        & \begin{tabular}[c]{@{}l@{}}134.7\\ ±10.8\%\end{tabular}          & \begin{tabular}[c]{@{}l@{}}221.6\\ ±8.9\%\end{tabular}  & \begin{tabular}[c]{@{}l@{}}258.5\\ ±13.4\%\end{tabular} & \begin{tabular}[c]{@{}l@{}}176.9\\ ±15.6\%\end{tabular}        & \begin{tabular}[c]{@{}l@{}}496.4\\ ±11.6\%\end{tabular} & \begin{tabular}[c]{@{}l@{}}116.1\\ ±14.2\%\end{tabular}                    \\ \midrule

\textbf{Total}                                                & \textbf{414.3 ms}                                                & \textbf{678.4 ms}                                       & \textbf{817.2 ms}                                       & \textbf{538.5 ms}                                              & \textbf{1429.8 ms}                                      & \textbf{330.6 ms}                                                          \\ \bottomrule
\end{tabular}

\caption{Speed comparison of JavaScript parsers.}
\label{table:speed-comparison-of-parsers}
\end{table}

\subsubsection{Metamodel and Precision}
For analysis and transformation purposes it is rather important to have a model with as much and as precise information as possible. If the parser produces a model conforming a detailed metamodel, it is easier to differentiate seemingly similar cases.

Based on the comparison~\cite{shift-spec-comparison} between the ESTree and the Shift AST specification, it is visible that Shift is a better choice if more detail is required.

% \subsubsection{Development Activity}
% \subsubsection{Platforms Supported}
